%\iffalse
%
%INSTRUCTIONS: (if this is not lecture1.tex, use the right file name)
%
%  Clip out the ********* INSERT HERE ********* bits below and insert
%appropriate TeX code.  Once you are done with your file, run
%
%  ``latex lecture1.tex''
%
%from a UNIX prompt.  If your LaTeX code is clean, the latex will exit
%back to a prompt.  Once this is done, run
%
%  ``dvips lecture1.dvi''
%
%which should print your file to the nearest printer.  There will be
%residual files called lecture1.log, lecture1.aux, and lecture1.dvi.
%All these can be deleted, but do not delete lecture1.tex.
%\fi
%%
\documentclass[11pt]{article}
\usepackage{amsfonts}
\usepackage{latexsym}
\setlength{\oddsidemargin}{.25in}
\setlength{\evensidemargin}{.25in}
\setlength{\textwidth}{6in}
\setlength{\topmargin}{-0.4in}
\setlength{\textheight}{8.5in}

\newcommand{\handout}[5]{
   \renewcommand{\thepage}{#1-\arabic{page}}
   \noindent
   \begin{center}
   \framebox{
      \vbox{
    \hbox to 5.78in { {\bf CSC 591 Cryptography} \hfill #2 }
       \vspace{4mm}
       \hbox to 5.78in { {\Large \hfill #5  \hfill} }
       \vspace{2mm}
       \hbox to 5.78in { {\it #3 \hfill #4} }
      }
   }
   \end{center}
   \vspace*{4mm}
}

\newcommand{\lecture}[3]{\handout{L#1}{#2}{Lecturer: Alessandra Scafuro}{Scribe: #3}{Lecture #1}}

\def\squarebox#1{\hbox to #1{\hfill\vbox to #1{\vfill}}}
\def\qed{\hspace*{\fill}
        \vbox{\hrule\hbox{\vrule\squarebox{.667em}\vrule}\hrule}}
\newenvironment{solution}{\begin{trivlist}\item[]{\bf Solution:}}
                      {\qed \end{trivlist}}
\newenvironment{solsketch}{\begin{trivlist}\item[]{\bf Solution Sketch:}}
                      {\qed \end{trivlist}}
\newenvironment{proof}{\begin{trivlist}\item[]{\bf Proof:}}
                      {\qed \end{trivlist}}

\newtheorem{theorem}{Theorem}
\newtheorem{corollary}[theorem]{Corollary}
\newtheorem{lemma}[theorem]{Lemma}
\newtheorem{observation}[theorem]{Observation}
\newtheorem{proposition}[theorem]{Proposition}
\newtheorem{definition}[theorem]{Definition}
\newtheorem{claim}[theorem]{Claim}
\newtheorem{fact}[theorem]{Fact}
\newtheorem{assumption}[theorem]{Assumption}

%Put more macros here, as needed.
\newcommand{\al}{\alpha}
\newcommand{\Z}{\mathbb Z}

\begin{document}

\lecture{CPA-Secure Encryption Algorithms}{9/26/18}{Bihan Zhang}


\section*{Background}

An encryption scheme is CPA secure if it is secure against chosen plaintext attacks.\\

The CPA game ($PrivK^{CPA}_{A,\Pi}(n)$) is as follows:

\hspace{\parindent} 1. Let $\Pi$ be an encryption oracle.
	 
\hspace{\parindent}	2. Let $A_{PPT}$ be an adversary
	
\hspace{\parindent}	3. Training Phase: $A_{PPT}$ can pass multiple messages $m_i$ to $\Pi$ and get the encrypted ciphertext $c_i$ back.
	
\hspace{\parindent}	4. Challenge Phase: 
	
\hspace{\parindent}\hspace{\parindent} a. $A_{PPT}$ passes $\{m_0, m1\}$ to $\Pi$. 

\hspace{\parindent}\hspace{\parindent} b. $\Pi$ picks either $m_0$ or $m_1$ and returns $c_b= Enc(m_b)$ 

\hspace{\parindent}\hspace{\parindent} c. $A_{PPT}$ guesses $b'$ ($b'$ is whichever $m_0$, or $m_1$ $A_{PPT}$ thinks $\Pi$ encrypted.

\hspace{\parindent}\hspace{\parindent} d. $A_{PPT}$ wins if $b==b'$ ($PrivK^{CPA}_{A,\Pi}(n)=1$)


Because $A_{PPT}$ has unlimited training time, if $\Pi$ is deterministic, there is a chance $\frac{g}{|M|}$, where g is the number of guesses $A_{PPT}$ has made in the training phase, that $A_{PPT}$ will win the CPA game. So for an encryption scheme to be considered CPA-secure it has to be deterministic.


-----------------------------------


A pseudorandom function (PRF) is a function whose output cannot be distinguished between a one chosen randomly by the PRF and a random oracle. A PRF guarantees all its outputs appears random (A PRG only guarantees a single output).

A function $F: \{0,1\}^* \times \{0,1\}^* \rightarrow \{0,1\}^*$ is considered pseudorandom if for all distinguishers $D_{PPT}$ the following holds true:\\

$|Pr[D^{F_k}\rightarrow 1] - Pr[D^{\{0,1\}^n}=1]| \leq neg(n)$\\


The PRF Game:\\

	\hspace{\parindent} 1. Let $F$ be an encryption oracle.
	
	\hspace{\parindent} 2. Let $D_{PPT}$ be a distinguisher with access to $F$.
	
	\hspace{\parindent} 3. $D_{PPT}$ passes $x_i$ to F
	
	\hspace{\parindent} 4. $F$ will either return $y_i = PRF(x_i)$ or $y_i = TR()$
	
	\hspace{\parindent} 5. $D_{PPT}$ guesses whether $y_i$ came from the PRF or from TR. $D_{PPT}$ wins if $|Pr[D^{F_k}\rightarrow 1] - Pr[D^{\{0,1\}^n}=1]| > neg(n)$\\
	

\section*{Topic/Problem}
This is the second lecture on CPA, with a focus on constructing a CPA-secure encryption scheme with a PRF.

\subsection*{Scheme}
Let $F$ be a pseudorandom function. Let $\Pi$ be (Gen, Enc, Dec) as described:\\

\hspace{\parindent} Gen: $k \epsilon \{0,1\}^n$\\

\hspace{\parindent} Enc(k, m): Given $k \epsilon \{0,1\}^n$ and $m \epsilon \{0,1\}^n$ choose a uniform $r \epsilon \{0,1\}^n$ and construct $c = <F_k(r)\oplus m, r>$\\

\hspace{\parindent} Dec(k,c): Given $k \epsilon \{0,1\}^n$ and $c=<s,r>$ construct $m= F_k(r)\oplus s$


\subsection*{Security Proof}

*********** WRITE THE PROOF. FIRST DESCRIBE AN HIGH-LEVEL INUTION OF WHY THE SCHEME IS SECURE, AND THE PROVIDE THE FORMAL PROOF (IF ANY) *************

Assume $F$ is a PRF, $\Pi$ is secure. \\

Intuition: Even though intuitively it feels insecure to reveal $r$ it actually doesn't matter. Even when $r$ is known, since $k$ is unknown, the result of $F_k(r)$ is indistinguishable from a random function. And because the output of $F_k(r)$ is pseudo-random, $F_k(r) \oplus m $ should be secure, and because $r$ is a random value each time, the scheme is non-deterministic, and should be CPA secure. \\




if $r$ $F_k(r)$

\end{document}
