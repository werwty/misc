% --------------------------------------------------------------
% This is all preamble stuff that you don't have to worry about.
% Head down to where it says "Start here"
% --------------------------------------------------------------

\documentclass[12pt]{article}

\usepackage[margin=1in]{geometry}
\usepackage{amsmath,amsthm,amssymb}
\usepackage{tikz}
\usepackage{mathtools}
\usepackage{listings}


\usepackage{graphicx}
\graphicspath{ {images/} }

\DeclarePairedDelimiter{\ceil}{\lceil}{\rceil}
\DeclarePairedDelimiter{\floor}{\lfloor}{\rfloor}

\usetikzlibrary{arrows}

\newcommand{\N}{\mathbb{N}}
\newcommand{\Z}{\mathbb{Z}}

\newenvironment{theorem}[2][Theorem]{\begin{trivlist}
		\item[\hskip \labelsep {\bfseries #1}\hskip \labelsep {\bfseries #2.}]}{\end{trivlist}}
\newenvironment{lemma}[2][Lemma]{\begin{trivlist}
		\item[\hskip \labelsep {\bfseries #1}\hskip \labelsep {\bfseries #2.}]}{\end{trivlist}}
\newenvironment{exercise}[2][Exercise]{\begin{trivlist}
		\item[\hskip \labelsep {\bfseries #1}\hskip \labelsep {\bfseries #2.}]}{\end{trivlist}}
\newenvironment{question}[2][Question]{\begin{trivlist}
		\item[\hskip \labelsep {\bfseries #1}\hskip \labelsep {\bfseries #2.}]}{\end{trivlist}}
\newenvironment{proposition}[2][Proposition]{\begin{trivlist}
		\item[\hskip \labelsep {\bfseries #1}\hskip \labelsep {\bfseries #2.}]}{\end{trivlist}}
\newenvironment{corollary}[2][Corollary]{\begin{trivlist}
		\item[\hskip \labelsep {\bfseries #1}\hskip \labelsep {\bfseries #2.}]}{\end{trivlist}}

\begin{document}
	
	% --------------------------------------------------------------
	%                         Start here
	% --------------------------------------------------------------
	
	%\renewcommand{\qedsymbol}{\filledbox}
	
	\title{Homework 3}%replace X with the appropriate number
	\author{Bihan Zhang - bzhang28 %replace with your name
		CSC 505} %if necessary, replace with your course title
	
	\maketitle
	
	
	\begin{question}{1d} 

		Algorithm:
		
		Build an array of size v+1. each element of the array keeps track of the optimal amount of coins it takes to construct it.
\begin{lstlisting}
def makeChange(denoms, value):
	
	# initialize array with infinity for all values
	dynamic_array = [float("inf") for i in range(0, value+1)]
	# except for the first value, 0. This is the base case:
	# it takes exactly 0 coins to create 0 value
	dynamic_array[0] = 0
			
	for i in range(1, value+1):
		for j in range(0, len(denoms)):
	
			if (denoms[j] <= i):
			  a = dynamic_array[i-denoms[j]]
			  if a < float("inf") and a + 1 < dynamic_array[i]:
			    dynamic_array[i]= a + 1
			
	return dynamic_array[value]; 
\end{lstlisting}
		
		For the inputs value=12, denoms=[1,6] on initialization
		
		\begin{lstlisting}
		dynamic_array = [0, inf, inf, inf, inf, inf, inf, 
					inf, inf, inf, inf, inf, inf]
		\end{lstlisting}
		
		After the first iteration we get:
		\begin{lstlisting}
		[0, 1, inf, inf, inf, inf, inf, inf, 
		inf, inf, inf, inf, inf]		
		\end{lstlisting}
		Since the only possible coin we can use for the first 5 iterations is the 1 cent coin, $dynamic\_array[i]$ just continues incrementing by one from $dynamic\_array[i-1]$.
		
		On the 6th iteration, we see the 6 cent coin come in, and we get:
		\begin{lstlisting}
		[0, 1, 2, 3, 4, 5, 1, inf, inf, inf, inf, inf, inf]		
		\end{lstlisting}
		
		For iteration 7-11, we continue incrementing by one from $dynamic\_array[i-1]$, until we finish at iteration 12, where we increment by one form $dynamic\_array[i-6]$, and we get:
		\begin{lstlisting}
		[0, 1, 2, 3, 4, 5, 1, 2, 3, 4, 5, 6, 2]		
		\end{lstlisting}
		

		
		
		Runtime: Since there are two loops, one of size n (for coin value), and one of size d (for coin denominations) The runtime of this is $\Theta(n*d)$ this can be treated the same as $\Theta(n)$ since d is of a fixed size. The space complexity of this algorithm is $O(n)$ since we only need on matrix of size n.
			\end{question}
	\begin{question}{3a}
		
		Take the substring x[i,...,j], if $x[i]==x[j]$ there is a palindrome of length $>=2$. If $x[i] != x[j]$, then we can consider instead the substrings x[i+1,....,j], x[i,....,j-1].
		
			$$L(i, j) = L (i+1, j-1) + 2 \text{ if } x[i]=x[j]$$
			$$L(i, j) = max(L(i+1, j), L(i,j-1 )) \text{ otherwise}$$
	
	\end{question}

	\begin{question}{3b}
Algorithm: Create a 2d array of size len(X) by len(X), each l[i][j] will store the longest palindrome in X[i...j]. 
Each element L[i][i] should be 1, since X[i...i] is a palindrome of 1.

Then for each substring, check if X[i] and X[j] are the same character, if they are, we know that the length of the longest substring is the length of L(X[i+1...j-1]) + 2.

If X[i] and X[j] are not equivalent, we know the longest substring is either in L(i+1, j) or L(i, j-1), whichever is greater.

		
		\begin{lstlisting}
def MPSP(X):
	l = [[0 for i in range(len(X))] for i in range(len(X))]
	
	# initialized l[i][i] with 1s, since each 
	# letter is a palindrome of length 1
	for i in range(len(X)):
		l[i][i]=1
	
	for s in range(1, len(X)):
		for i in range(len(X)-s):         
			j= i+s
			if X[i] == X[j]:
				l[i][j] = l[i+1][j-1]+2
			else:
				l[i][j] = max(l[i+1][j], l[i][j-1])
	return l[0][len(X)-1]
					
					
		\end{lstlisting}
		
		
		Let's work through an example of running MPSP on the string $asdfda$
		
		The L matrix looks like the following on initialization, because $L(0,0)=1$, $L(1,1)=1$, and so on.
		\begin{lstlisting}
		1 0 0 0 0 0
		0 1 0 0 0 0
		0 0 1 0 0 0
		0 0 0 1 0 0
		0 0 0 0 1 0
		0 0 0 0 0 1
		\end{lstlisting}
		
		On the second iteration we look at the substring $'as'$, this isn't a palindrome, and so we fill the matrix $L[0][1]$ with 1, since the longest palidrome is still 1
		
		\begin{lstlisting}
		1 1 0 0 0 0
		0 1 0 0 0 0
		0 0 1 0 0 0
		0 0 0 1 0 0
		0 0 0 0 1 0
		0 0 0 0 0 1
		\end{lstlisting}
		
		This continues on for a while until we get to i=2 j=4, or the substring 'dfd', since $X[i]==X[j]$ we check $L(3,3)$ and increment this by 2 to get L(2,4)=3. 
		
		\begin{lstlisting}
		1 1 1 0 0 0
		0 1 1 1 0 0
		0 0 1 1 3 0
		0 0 0 1 1 0
		0 0 0 0 1 1
		0 0 0 0 0 1
		\end{lstlisting}
		
		This continues on until the program concludes and we see the longest palindrome is 5:
		\begin{lstlisting}
		1 1 1 1 3 5
		0 1 1 1 3 3
		0 0 1 1 3 3
		0 0 0 1 1 1
		0 0 0 0 1 1
		0 0 0 0 0 1
		\end{lstlisting}
		
		
				
		The runtime of this algorithm is equivalent to $\sum_{i=1}^{n}{\sum_{s=0}^{n-s}{1}}$ which is in $O(n^2)$


	\end{question}

	\begin{question}{4}
		Place each tower $y_j$ as far from the houses as possible while maintaining the property that there is a tower within distance d. Start by placing the first tower $y_0$ at $x_0 + d$. Then for each subsequent house, check if the last tower is within distance d. If the last tower is not within distance d, then place the next tower $x_n+d$ away from that house.
		\begin{lstlisting}
def place_towers(d, x):
	y[0] = x[0] + d
	j=0
	
	for i in range(1,len(x)):
		if x[i]-y[j] > d:
			j+=1
			y[j] = x[i]+d
\end{lstlisting}


Let's work through and example of this with $d=2$ and $x=[0,1,2,3,4,5,6]$

We place the first tower at $x_0 + d$, so $y=[2]$.
Continuing down the $x$ list 1 is within range, 2 is within range, so is 3 and 4. 
5 is not within range so we add a tower at $5+2=7$, $y=[2,7]$. 6 is also within range of this. So we end up with $y=[2,7]$.

This algorithm runs in $O(n)$ time, we only run through n comparisons in the loop. \\

Let O be the optimal solution to the tower placing problem, that places $z$ towers at distances $d_1<d_2<...d_z$ none of which overlap.

Let PT be our solution as described by the above algorithm, where we place $j$ towers at distances $y_1<y_2< ... y_j$.\\

Claim1: $z\geq j$ (the number of towers allocated by the optimal solution is greater than or equal to the number of towers allocated by PT)\\

Claim2: $d_z\leq y_j$ (The tower placed by the optimal solution is not greater than the tower placed by PT)


In the trivial case of 0 houses, O and PT both generate 0 towers, which is equivalent.

In the case of 1 house, we can prove O is equivalent to PT via contradiction.

Suppose for the sake of contradiction that Claim1 and Claim2 are false, then either $z=0$ or $d_1 > y_1$ have to be true. Since we're assuming that there is a house and O is a solution, the first house must get some coverage from the tower. $z$ must be 1, so $d_1 > y_1$ must be true. Since $y_1$ was placed at $x_1+d$, $d_1$ being greater means that $x_1$ gets no coverage from $d_1$. This is a contradiction, thus Claim1 and Claim2 must be true for a single house.


In the case of more than one house we can prove O is equivalent to PT via more contradiction.
Suppose for the sake of contradiction that Claim1 and Claim2 are false, let k be the smallest index for which Claim1 and Claim2 fails. 

For the index (k-1) Claim1 and Claim2 is true. So we know $d_{k-1} \leq y_{k-1}$. 
But for index k we assume $d_k > y_k$ since we know $y_k - y{k-1} = d$ there is some space between $d_{k-1}$ and $d_k$ will not be covered. so some $x_{k-1/2}$ will not be covered by O.
This leads to a contradiction, therefore Claim1 and Claim2 must be true for all houses, and O is equivalent to PT.


		
	\end{question}

	\begin{question}{5}
		
		Step 1: MAKE-SET each of $\{x_1\},..., \{x_n\}$.\\
		Step 2: Union each of $x_{n-1}, x_n$ e.g. UNION($x_1$,$x_2$) UNION($x_3$,$x_4$)\\
		Step 3: Union each of these resulting sets  UNION(UNION($x_1$,$x_2$), UNION($x_3$,$x_4$)\\
		This results in binomial tree that has ${k \choose i}$ nodes at depth $i$ ($2^k$ node total). At least half of the $n$ nodes are at depth $log(n)/2$ and lower. 
		So running FIND-SET should take $\Omega(log(n))$ time. Assume $m$ is large ($m-2n$ is runtime $\Omega(n)$). $\Omega((m-2n)log(n)) = \Omega(mlog(n))$.
	\end{question}
	\clearpage
	
	
	% --------------------------------------------------------------
	%     You don't have to mess with anything below this line.
	% --------------------------------------------------------------
	
\end{document}