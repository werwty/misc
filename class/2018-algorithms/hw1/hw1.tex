% --------------------------------------------------------------
% This is all preamble stuff that you don't have to worry about.
% Head down to where it says "Start here"
% --------------------------------------------------------------

\documentclass[12pt]{article}

\usepackage[margin=1in]{geometry}
\usepackage{amsmath,amsthm,amssymb}
\usepackage{tikz}
\usepackage{mathtools}
\usepackage{listings}


\usepackage{graphicx}
\graphicspath{ {images/} }

\DeclarePairedDelimiter{\ceil}{\lceil}{\rceil}
\DeclarePairedDelimiter{\floor}{\lfloor}{\rfloor}

\usetikzlibrary{arrows}

\newcommand{\N}{\mathbb{N}}
\newcommand{\Z}{\mathbb{Z}}

\newenvironment{theorem}[2][Theorem]{\begin{trivlist}
		\item[\hskip \labelsep {\bfseries #1}\hskip \labelsep {\bfseries #2.}]}{\end{trivlist}}
\newenvironment{lemma}[2][Lemma]{\begin{trivlist}
		\item[\hskip \labelsep {\bfseries #1}\hskip \labelsep {\bfseries #2.}]}{\end{trivlist}}
\newenvironment{exercise}[2][Exercise]{\begin{trivlist}
		\item[\hskip \labelsep {\bfseries #1}\hskip \labelsep {\bfseries #2.}]}{\end{trivlist}}
\newenvironment{question}[2][Question]{\begin{trivlist}
		\item[\hskip \labelsep {\bfseries #1}\hskip \labelsep {\bfseries #2.}]}{\end{trivlist}}
\newenvironment{proposition}[2][Proposition]{\begin{trivlist}
		\item[\hskip \labelsep {\bfseries #1}\hskip \labelsep {\bfseries #2.}]}{\end{trivlist}}
\newenvironment{corollary}[2][Corollary]{\begin{trivlist}
		\item[\hskip \labelsep {\bfseries #1}\hskip \labelsep {\bfseries #2.}]}{\end{trivlist}}

\begin{document}
	
	% --------------------------------------------------------------
	%                         Start here
	% --------------------------------------------------------------
	
	%\renewcommand{\qedsymbol}{\filledbox}
	
	\title{Homework 1}%replace X with the appropriate number
	\author{Bihan Zhang - bzhang28 %replace with your name
		CSC 505} %if necessary, replace with your course title
	
	\maketitle
	
	
	\begin{question}{1a} 
	$\Theta(n)$
	Reasoning: $y = a_i+ x  y$ gets executed once per loop iteration. So total number of executions is n, since the look executes 0-n times. 		
	\end{question}

	\begin{question}{1b} 
	\begin{lstlisting}
	
		y=0
		for i= range(0, n)
		m= 1
		for j= range(1, i)
		m= m*x
		y= a_i*m + y 
		
	\end{lstlisting}
	
	The running time of this is $\Theta(n^2)$ since there is one nested loop. Compared to Horner's rule the running time is longer.

	\end{question}
		
	\begin{question}{1c} 
	Initialization: At the start of the loop i=n. 

	$$y=\sum_{k=0}^{n-(n+1)}a_{k+n+1}x^k = \sum_{k=0}^{-1}a_{k+n+1}x^k = 0 $$ So the loop invariant holds prior to the first loop\\
	
	Maintenance: At an arbitrary ith iterations where $0<=i<n$.
	$$y=a_i+x*y = a_i+x*\sum_{k=0}^{n-(i+1)}a_{k+i+1}x^k$$
	$$=a_{i}x^{0}+\sum_{k=0}^{n-(i+1)}a_{k+i+1}x^{k+1}$$
	$$=a_{i}x^{0}\sum_{k=1}^{n-i}a_{k+i}x^{k}$$	
	$$=\sum_{k=0}^{n-i}a_{k+i}x^{k}$$
	
	The loop invariant holds.
	
	Termination: The loops terminate when $i=-1$:
	$$y=\sum_{k=0}^{n-(-1)-1}a_{k-1+1}x^{k}=\sum_{k=0}^{n}a_{k}x^{k}$$
	The invariant still holds, hence the algorithm is correct.
	\end{question}	

	\begin{question}{2a} 		
		I assumed that an operation like $k \leftarrow i*i*i+2$ uses two multiplications and one addition	
		
		\begin{tabular}{|l|l|l|l|l|l|}
			\hline
			\textbf{n}                       & \textbf{1} & \textbf{2} & \textbf{3} & \textbf{4} & \textbf{5} \\ \hline
			\textit{\textbf{return value k}} & 3          & 3          & 3          & 10         & 10         \\ \hline
			\textit{\textbf{return value l}} & .5         & .5         & .75        & 1.5        & 3.75       \\ \hline
			\textbf{multiplications (*)}     & 4          & 6          & 8          & 12         & 14         \\ \hline
			\textbf{additions (+)}           & 1          & 1          & 1          & 2          & 12         \\ \hline
		\end{tabular}
		
	\end{question}

	\begin{question}{2b}	
		b) (2 points) As a function of n, what is the value of k returned in line 7? Justify your results.
	\end{question}

	\begin{question}{2c}	
		c) (2 points) As a function of n, what is the value of l returned in line 7? Justify your results.
		\end{question}

\begin{question}{2d}	
		d) (2 points) As a function of n, how many multiplications (“*”) does the algorithm perform? Justify your results.
			\end{question}
	
\begin{question}{2e}	
		e) (2 points) As a function of n, how many additions (“+”) does the algorithm perform? Justify your results.
			\end{question}
	
\begin{question}{3a}


$n^{1/2}+999/n$ $\epsilon$ $O(lg(n)+n)$\\
Justification: $\lim_{n\to\infty} (sqrt(n) + 999/n)/(lg(n)+n) = 0$\\

$ln(n)+n$ $\epsilon$ $O(n^{.99}lg(n))$ \\
Justification: $\lim_{n\to\infty} (ln(n)+n)/(n^{.99}lg(n)) = 0$\\

$n^{.99}lg(n)$ $\epsilon$ $O(ln(n!))$ \\
$ln(n!)$* $\epsilon$ $\Theta(nlg(n))$\\
$nlg(n)$ $\epsilon$ $O(n^{1.01})$\\




lg(n)+ n,	lg(n!),	n0.99lg(n),	nlg(n),	n1/2+999/n,	n1.01,		999n1/2


			\end{question}
	
\begin{question}{3b}	
		b) lg(n!),	100!,	n1.01,	2lg(n)+lg(n)+n1/3,		n1/3,	1+1/n,	 	lg(nn)
			\end{question}
	
\begin{question}{4a}	
		4. (12 points) Prove or disprove rigorously (ie give values for c and n0 that will make your argument work) using the formal definitions of , ω (little-omega), and o (little-oh).
		a) (4 points) nlg(n) + n - 1000  ω( n )  (little-omega).
			\end{question}
	
\begin{question}{4b}	
		b) (4 points) nlg(n) + n0.5 + n-1  ( nlg(n) ).
			\end{question}
	
\begin{question}{4c}	
		c) (4 points) lg(n) o( ln(n3) ) (little-oh).
		
			\end{question}
	
\begin{question}{5a}	
		5. (6 points) Purpose: Practice proving asymptotic relationships. In proving big-oh and big-omega bounds there is a relationship between the c that is used and the smallest n0 that will work (for O, the smaller the c, the larger the n0; for big-omega, the larger the c, the larger the n0). In each of the following situations, describe (the smallest integer) n0 as a function of c. You'll have to use the ceiling function to ensure that n0 is an integer. Your solution should also give you a lower bound (for big-oh) or an upper bound (for big-omega) on the constant c.
		(a) (3 points) Let f(n) = 5n4 - 7n3 and prove that f(n)  O(n4).
			\end{question}
	
\begin{question}{5b}	
		(b) (3 points) Let f(n) = 2n4 + 4n3 and prove that f(n)  (n3).
			\end{question}
	
\begin{question}{6}	
		6. (6 points) Purpose: Practice how to design, analyze, and communicate algorithms. Describe an non-recursive Θ(lg n) algorithm which computes (a/2)2n, given a and n. You may assume that a is a positive real number, and n a positive integer, but do not assume that n is a power of 2. Please follow the above instructions for describing your algorithm. Please give both, a textual description and pseudocode of your algorithm, and justify the asymptotic running time of your algorithm.
		
		
	\end{question}
	
	\clearpage
	
	
	% --------------------------------------------------------------
	%     You don't have to mess with anything below this line.
	% --------------------------------------------------------------
	
\end{document}