% --------------------------------------------------------------
% This is all preamble stuff that you don't have to worry about.
% Head down to where it says "Start here"
% --------------------------------------------------------------

\documentclass[12pt]{article}

\usepackage[margin=1in]{geometry}
\usepackage{amsmath,amsthm,amssymb}
\usepackage{tikz}
\usepackage{mathtools}
\usepackage{listings}


\usepackage{graphicx}
\graphicspath{ {images/} }

\DeclarePairedDelimiter{\ceil}{\lceil}{\rceil}
\DeclarePairedDelimiter{\floor}{\lfloor}{\rfloor}

\usetikzlibrary{arrows}

\newcommand{\N}{\mathbb{N}}
\newcommand{\Z}{\mathbb{Z}}

\newenvironment{theorem}[2][Theorem]{\begin{trivlist}
		\item[\hskip \labelsep {\bfseries #1}\hskip \labelsep {\bfseries #2.}]}{\end{trivlist}}
\newenvironment{lemma}[2][Lemma]{\begin{trivlist}
		\item[\hskip \labelsep {\bfseries #1}\hskip \labelsep {\bfseries #2.}]}{\end{trivlist}}
\newenvironment{exercise}[2][Exercise]{\begin{trivlist}
		\item[\hskip \labelsep {\bfseries #1}\hskip \labelsep {\bfseries #2.}]}{\end{trivlist}}
\newenvironment{question}[2][Question]{\begin{trivlist}
		\item[\hskip \labelsep {\bfseries #1}\hskip \labelsep {\bfseries #2.}]}{\end{trivlist}}
\newenvironment{proposition}[2][Proposition]{\begin{trivlist}
		\item[\hskip \labelsep {\bfseries #1}\hskip \labelsep {\bfseries #2.}]}{\end{trivlist}}
\newenvironment{corollary}[2][Corollary]{\begin{trivlist}
		\item[\hskip \labelsep {\bfseries #1}\hskip \labelsep {\bfseries #2.}]}{\end{trivlist}}

\begin{document}
	
	% --------------------------------------------------------------
	%                         Start here
	% --------------------------------------------------------------
	
	%\renewcommand{\qedsymbol}{\filledbox}
	
	\title{Homework 1}%replace X with the appropriate number
	\author{Bihan Zhang - bzhang28 %replace with your name
		CSC 505} %if necessary, replace with your course title
	
	\maketitle
	
	
	\begin{question}{1a} 
	$\Theta(n)$
	Reasoning: $y = a_i+ x  y$ gets executed once per loop iteration. So total number of executions is n.		
	\end{question}

	\begin{question}{1b} 
	\begin{lstlisting}
	
		y = 0
		# the +1 in range is a python thing
		for i in range(0, n+1)
			m = 1
			for j in range(1, i+1)
				m = m*x
			y = a_i*m + y 
		
	\end{lstlisting}
	
	The running time of this is $\Theta(n^2)$ since there is one nested loop in the main loop. Compared to Horner's rule the running time is longer.

	\end{question}
		
	\begin{question}{1c} 
	Initialization: At the start of the loop i=n. 

	$$y=\sum_{k=0}^{n-(n+1)}a_{k+n+1}x^k = \sum_{k=0}^{-1}a_{k+n+1}x^k = 0 $$ So the loop invariant holds prior to the first loop\\
	
	Maintenance: At an arbitrary ith iterations where $0<=i<n$.
	$$y=a_i+x*y = a_i+x*\sum_{k=0}^{n-(i+1)}a_{k+i+1}x^k$$
	$$=a_{i}x^{0}+\sum_{k=0}^{n-(i+1)}a_{k+i+1}x^{k+1}$$
	$$=a_{i}x^{0}\sum_{k=1}^{n-i}a_{k+i}x^{k}$$	
	$$=\sum_{k=0}^{n-i}a_{k+i}x^{k}$$
	
	The loop invariant holds.
	
	Termination: The loops terminate when $i=-1$:
	$$y=\sum_{k=0}^{n-(-1)-1}a_{k-1+1}x^{k}=\sum_{k=0}^{n}a_{k}x^{k}$$
	The invariant still holds, hence the algorithm is correct.
	\end{question}	

	\begin{question}{2a} 		
		I assumed that an operation like $k \leftarrow i*i*i+2$ uses two multiplications and one addition	
		
		\begin{tabular}{|l|l|l|l|l|l|}
			\hline
			\textbf{n}                       & \textbf{1} & \textbf{2} & \textbf{3} & \textbf{4} & \textbf{5} \\ \hline
			\textit{\textbf{return value k}} & 3          & 3          & 3          & 10         & 10         \\ \hline
			\textit{\textbf{return value l}} & .5         & .5         & .75        & 1.5        & 3.75       \\ \hline
			\textbf{multiplications (*)}     & 4          & 6          & 8          & 12         & 14         \\ \hline
			\textbf{additions (+)}           & 1          & 1          & 1          & 2          & 12         \\ \hline
		\end{tabular}
		
	\end{question}

	\begin{question}{2b}	
		b) (2 points) As a function of n, what is the value of k returned in line 7? Justify your results.
	\end{question}

	\begin{question}{2c}	
		c) (2 points) As a function of n, what is the value of l returned in line 7? Justify your results.
		\end{question}

\begin{question}{2d}	
		d) (2 points) As a function of n, how many multiplications (“*”) does the algorithm perform? Justify your results.
			\end{question}
	
\begin{question}{2e}	
		e) (2 points) As a function of n, how many additions (“+”) does the algorithm perform? Justify your results.
			\end{question}
	
\begin{question}{3a}
		
		Sorting from lowest Asymptotic growth to the highest:\\
		
		$n^{1/2}+999/n$ $\epsilon$ $o(n^{.99}lg(n))$\\
		Justification: $\lim_{n\to\infty} (sqrt(n) + 999/n)/(n^{.99}lg(n)) = 0$\\
		
		$n^{.99}lg(n)$ $\epsilon$ $o(lg(n)+n)$\\
		Justification: $\lim_{n\to\infty} (n^{.99}lg(n))/(lg(n)+n) = 0$\\
		
		$lg(n)+n$ $\epsilon$ $o(n*lg(n))$ \\
		Justification: $\lim_{n\to\infty} (lg(n)+n)/(n*lg(n)) = 0$\\
				
		$ln(n!)$* $\epsilon$ $\Theta(n*lg(n))$\\
		Justification: According to Stirling's Approximation $ln(n!)$ $\epsilon$ $\Theta(n*lg*n)$\\
		
		$n*lg(n)$ $\epsilon$ $o(n^{1.01})$\\
		Justification: $\lim_{n\to\infty} (n*lg(n))/(n^{1.01}) = 0$\\
	
\end{question}
	
\begin{question}{3b}	
	
	Sorting from lowest Asymptotic growth to the highest:\\
			
	$100!$ $\epsilon$ $o(1+1/n)$\\
	Justification: $100!$ is a constant, and grows smaller than anything with $n$\\
	
	$1+1/n$ $\epsilon$ $o(n^{1/3})$\\
	Justification: $\lim_{n\to\infty} (1+1/n)/(n^{1/3}) = 0$\\
			
	$n^{1/3}$* $\epsilon$ $\Theta(2^lg(n)+lg(n)+n^{1/3})$\\
	Justification: $\lim_{n\to\infty} (n^{1/3})/(2^lg(n)+lg(n)+n^{1/3}) = 1$\\
				
	$2^lg(n)+lg(n)+n^{1/3}$ $\epsilon$ $o(lg(n!))$\\
	Justification: $\lim_{n\to\infty} (2^lg(n)+lg(n)+n^{1/3})/(lg(n!)) = 0$\\
	
	$lg(n!)$* $\epsilon$ $\Theta(lg(n^n))$ \\
	Justification: $\lim_{n\to\infty} (lg(n!))/(lg(n^n)) = 1$\\
		
	$lg(n^n)$ $\epsilon$ $o(n^{1.01})$\\
	Justification: $\lim_{n\to\infty} (lg(n^n))/(n^{1.01}) = 0$\\
	
			\end{question}
	
\begin{question}{4a}	
$n*lg(n)+n-1000$ $\epsilon$ $\omega(n)$\\

By definition of $\omega$:

$f(n) \epsilon \omega(g(n))$ iff $\forall{c>0}$ $\exists{n_0>0}$ such that $f(n) > cg(n) \geq 0$\\

$$n*lg(n)+n-1000 > cn \geq 0$$

$$lg(n)+1-1000/n > c$$

This is true for $n_0>1000$ and $c<lg(n)$. So the asymptotic relationship $n*lg(n)+n-1000$ $\epsilon$ $\omega(n)$ does hold true. 

			\end{question}
	
\begin{question}{4b}	
		$n*lg(n) + n^{0.5} + n^{-1}$ $\epsilon$ $\Theta( n*lg(n) )$
		
		To show that this is true, we must show that:\\
				$n*lg(n) + n^{0.5} + n^{-1}$ $\epsilon$ $O( n*lg(n) )$
				and\\
					$n*lg(n) + n^{0.5} + n^{-1}$ $\epsilon$ $\Omega( n*lg(n) )$\\
						
						
		To prove $Big-O$:\\
		$f(n) \epsilon O(g(n))$ iff $\exists c, n_0 > 0$ s.t. $0\leq f(n) \leq cg(n)$\\
		restrict $n\geq1$
			$$n*lg(n) + n^{0.5} + n^{-1} \leq cn*lg(n)$$
			$$lg(n) + n^-.5 + n^{-2} \leq c*lg(n)$$\\
			restrict $lg(n)>1$ or $n_0>2$\\
			For $c>0$ this inequality holds.  So 				$n*lg(n) + n^{0.5} + n^{-1}$ $\epsilon$ $O( n*lg(n) )$ with
			$c=1$ and $n_0=2$\\
			
	
		To prove $Big-\Omega$:\\
			$f(n) \epsilon \Omega(g(n))$ iff $\exists c, n_0 > 0$ s.t. $f(n) \geq cg(n) \geq 0$\\

			$$n*lg(n) + n^{0.5} + n^{-1} \geq cn*lg(n) \geq 0$$\\
			restrict $n>1$
			$$lg(n)+n^{-.5}+n^{-2} \geq clg(n) \geq 0$$
			restrict $lg(n)>1$ or $n_0>2$\\
			This inequality holds true for $0<c<1$ so choose c to be .5\\
			$n*lg(n) + n^{0.5} + n^{-1}$ $\epsilon$ $\Omega( n*lg(n) )$\\ holds for $c=.5$ $n_0=2$\\
			
			Since $f(n) \epsilon O(g(n))$ and $f(n) \epsilon \Omega(g(n))$, it must be true that $f(n) \epsilon \Theta(g(n))$
			

			\end{question}
	
\begin{question}{4c}	
		$lg(n)$ $\epsilon$ $o(ln(n^3))$\\
		Proof via contradiction:\\
		
		For the above to hold:\\
		$\forall{c>0}$ $ \exists{n_0>0}$ s.t $ 0\leq f(n) \leq c*g(n)$
		
		$$0\leq lg(n) \leq c*ln(n^3)$$
		$$0\leq lg(n) \leq 3c*lg(n)$$
		
		For $0<c<1/3$ there is no n such that $lg(n) \leq 3c*lg(n)$
		Therefore 		$lg(n)$ is not characterized by $o(ln(n^3))$\\
		
			\end{question}
	
\begin{question}{5a}	
	
		To prove a Big-O relationship:\\
		$f(n) \epsilon O(g(n))$ iff $\exists{c, n_0 >0}$ s.t. $0 \leq f(n) \leq cg(n)$
		$$0 \leq 5n^4 - 7n^3 \leq cn^4$$
		$$5-7/n \leq c$$
		$$n\geq -7/(c-5)$$
		$$n_0(c)=\ceil{-7/(c-5)}$$

		
			\end{question}
	
\begin{question}{5b}
	To prove a Big-$\Omega$ relationship:\\
	
	$f(n) \epsilon \Omega(g(n))$ iff $\exists{c,n_0}>0$ s.t. $f(n) \geq cg(n) \geq 0$
	$$ 2n^4+4n^3 \geq cn^3 \geq 0$$
	$$2n+4 \geq c$$
	$$n_0(c)=\ceil{c/2-2}$$
	
			\end{question}
	
\begin{question}{6}	

	Consider an exponent $x^n$. We can write $n$ as a series of binary bits: $n=b_0b_1b_2...b_i$
	
	The exponent can then be written as $x^{b_0b_1b_2...b_i} = x^{b_0}*x^{b_1}...*x^{b_i}$
	

	If we compute the exponent by halving
	$n$ and doubling $x$ every step, we can achieve this calculation in less time than $\Theta(n)$
	
	If $n$ is odd at any step, we store an extra variable $y$ that contains the product of all the times an odd exponent occurred. After storing this value half the exponent ($\floor{n/}$), and square the base.
	
	This value $y$ gets multiplied to the $x$ value one more time before returning. 
		
	Since we half $n$ each time, the computation time of this algorithm is $\Theta(lg(n))$	
	\begin{lstlisting}
	
	x = a/2
	e = 2*n	
	y = 1
	
	while e > 1:
		if e%2==0: 
			x = x*x
			e = e/2
		else:
			y = x*y
			x = x*x
			e = (e-1)/2
	return x*y
	\end{lstlisting}
	
	Worked Example:
		\begin{lstlisting}
		Say a=14 and n=150
				
		1st loop, e is 150.0, x is 49.0, y is 1.
		2nd loop, e is 75.0, x is 2401.0, y is 1.
		3rd loop, e is 37.0, x is 5764801.0, y is 2401.0.
		4th loop, e is 18.0, x is 33232930569601.0, y is 13841287201.0.
		5th loop, e is 9.0, x is 1.1044276742439207e+27, y is 13841287201.0.
		6th loop, e is 4.0, x is 1.2197604876358358e+54, y is 1.5286700631942576e+37.
		7th loop, e is 2.0, x is 1.4878156471976118e+108, y is 1.5286700631942576e+37.
		8th loop, e is 1.0, x is 2.2135954000460487e+216, y is 1.5286700631942576e+37.
		return value is 3.3838570200749116e+253
		\end{lstlisting}
		
	\end{question}
	
	\clearpage
	
	
	% --------------------------------------------------------------
	%     You don't have to mess with anything below this line.
	% --------------------------------------------------------------
	
\end{document}